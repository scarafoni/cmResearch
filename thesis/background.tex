%background

\section{Crowdsourcing}
Crowdsourcing has shown itself to be a very effective computational resource.
Formally, it is defined as
\begin{quote}
the act of a company or institution taking a function once performed by employees and outsourcing it to an undefined (and generally large) network of people in the form of an open call. \cite{brabham2008crowdsourcing}.
\end{quote}

\subsection{Current Scope and Usage of Crowdsourcing}
Though crowdsourcing covers a very broad range of activities and fields, it is possible to define separate catagories. Brabham defined three separate categories:
\begin{enumerate}
	\item crowdfunding
	\item crowd labor 
	\item crowd research.
\end{enumerate} 

Crowdfunding is the use of crowd-based resources to fund projects. Websites such as gofundme.com, kickstarter, and patreon.com are all crowdfunding website which allow elicit funds from users. 
This method is particularly useful for freelance artists \cite{brabham2008crowdsourcing}.

Crowd labor utilizes the crowd for computation and other work-like tasks. Generally, this involves the users doing low effort tasks for low pay. 
In many situations, this can result in a much cheaper product than non crowdsourced, professional settings.
Crowdsourcing is also often used for tasks which are prohibitively difficult or impossible with current computational means.

For example, the stock photo website iStockphoto collects images uploaded by users, and pays them a small compensation for their effort. 
This website is able to lease the rights to these images for \$1, which undercuts professional photographers' prices by over $90\%$, Crowd labor also extends to a number of other fields, including industrial research \cite{howe2006rise}.

\subsection{Effective Crowdsourcing}



% tips for crowd motivaiton: 4 f's fun, fulfillment, fame, fortune
%crowdsourcing
%crowd sourcing in general

%redundant sentence removal
%sentence comparisons
%which methods work best and why
%semantic distance between sentences
%hierarchical clusting of sentences
%unsupervised learning of setnences
