%Introduction
Crowdsourcing has shown to be an effective means of solving tasks that are computationally difficult, or otherwise impractical, for artificial intelligence programs to solve. \cite{howie2006rise}. 
There are many instances, of course,where humans are more effective than machines and vice-versa. 
To pick the most obvious example: humans are more efficient at qualitative, ambiguous tasks, while artificial intelligence (AI) agents tend to be better at more quantitative, direct tasks.
Naturally, most tasks do not fall cleanly into one category or another; there are many tasks which cannot be done perfectly by human or AI agents. 
For this reason, the merging of human and AI is necessary for the completion of a number of tasks.

Consider, for example, the task of building an instruction manual for a common profession. 
If a restaurant wanted to create a manual for a number of everyday tasks in the restaurant in order to train new employees quickly.
\todo{ADD SOMETHING ABOUT THIS BEING ABOUT SKILLED USERS}

It would use data gathered from existing employees to solve this task, as the ambiguous and qualitative data of procedural tasks such making food and settling tables is far beyond the current capabilities of artificial intelligence.
One experienced employee could be delegated to the task-instruction creation, but there are a number of issues with this method.
First, an over-reliance on one individual leaves the system vulnerable to error.
An individual may make mistakes when writing instructions.
The only solution to this problem would be to have other users check the first users work, which automatically means that multiple users will be needed regardless.
Second, many task, especially the qualitative, ambiguous ones previously mentioned, can be done in multiple ways.
While the differences may not necessarily be erroneous in nature, they can pose problems.
Certain delis, for example, have managers with different policies on the quality of meat used for sandwiches.
If a trainee is taught the preference of one manager and not the other, issues can arise if the employee switches managers.
Third, there are many instances where individuals have been shown to be slower than crowds \cite{lasecki2013interactive,}.
It it likely that crowd-sourced users will also be more efficient than single users in this domain.

One major issue with gathering instructions from users is dealing with redundancy.
Because many people have similar notions of instructions for various tasks, users tend to give similar instructions for a given step.
This leads to redundancy. For example, if a group of 3 users were asked to write the first step for making a peanut-butter and jelly sandwich, the users might respond with
\begin{enumerate}
\item get two slices of bread
\item get two slices of bread, a knife, peanut butter, and some jelly
\item get the ingredients.
\end{enumerate}
It is obvious in this situation that because suggestion 3 is described completely (and more thoroughly) by suggestion 2, that it is redundant and thus can be removed.

Failure to do so can impede the crowd-sourcing process.
First, it introduces a form of noise into the system.
With multiple identical sentences, if users are asked to vote on the suggestions, the voting process may be subverted.
Consider the same situation as the last example, but the three inputs are
\begin{enumerate}
\item get two slices of bread
\item get the ingredients
\item get the ingredients.
\end{enumerate}
If two users believe that option 2 (or equivalently, option 3) is correct, they could vote for either. 
If the third user votes for option 1, then there is no clear majority from voting, even though the consensus is clear.

This problem ultimately is an issue of natural language processing, and is unique in that it deals with very short documents in a very small corpus. 
A quick and efficient method of removing and/or minimizing redundancies in the context of this problem is essential to the completion of the task.

In the simply sense, the task it to take a series of unlabeled sentences and break them into groups depending on similarity.
This naturally frames the problem as an unsupervised learning problem.
More specifically, it is a problem that can be solved with a clustering algorithm.
Further, there are multiple means of catagorizing sentences. 
Returning to the first list of sentences, it could be argued that setnence one, ``get two slices of bread,'' and sentence two, ``get two slices of bread, a knife, peanut butter, and some jelly'' should be merged because the former is a subset of actions of the latter.
However, it could also be argued that the two should not be merged because they ultimately display two distinct thoughts.
It is therefore good to find a clustering system that allows for grades or shades of similarity classiffication.
Because this is one of the core features of agglomerative  hierarchical clustering, it is hypothesized that this form of clustering will provide the best model for solving the problem.

In total, the task of building a set of instructions (or a ``how-to'' manual) efficiently from a group of users requires efficient crowd-sourcing techniques as well as sufficiently sophisticated AI to augment this process.
Although there has been much research on using crowds to answer single questions, there has not been much work done in the next level of abstraction: having users answer lists of questions (i.e. instructions) \cite{lasecki2013chorus,bigham2010vizwiz}.

***add objectives, purpose, etc***

The remainder of this paper will be organized as follows:
\begin{itemize}
	\item the Background section will discuss previous work done in both crowdsourcing and redundant sentence removal, as well the unresearched areas where this project will make its contributions
	
	\item The Methodology section will describe the Canned Mentorship system in detail, both at an algorithmic and implementation level. It will also descirbe the parameters tested and the experimental setups.
	
	\item The results section will describe the experimental results of the study, and discuss the any anomalies or other observations encountered during the testing process.
\end{itemize}
